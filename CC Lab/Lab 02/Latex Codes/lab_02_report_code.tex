\documentclass[12pt]{article}

% Page setup
\usepackage[a4paper, margin=1in]{geometry}

% Font
\usepackage{mathpazo}

% Formatting packages
\usepackage{setspace}
\usepackage{graphicx}
\usepackage{listings}
\usepackage{titlesec}

% Remove page numbers
\pagenumbering{gobble}

% Line spacing
\onehalfspacing

% Code style
\lstset{
    basicstyle=\ttfamily\small,
    frame=single,
    breaklines=true,
    tabsize=2
}

% Section formatting
\titleformat{\section}{\large\bfseries}{\thesection}{1em}{}

\begin{document}
\vspace{1cm}

\section*{Experiment No 02}

\section*{Experiment Name}
A program to identify whether a given line is a comment or not and generate a comment-free C program.

\section*{Objective}
The objective of this experiment is to design and implement a basic lexical analyzer to detect comments in a C program. The program identifies both single-line and multi-line comments, displays the line numbers where comments occur, and generates a new file containing the source code without comments. This experiment helps in understanding lexical analysis concepts used in compiler design.

\section*{Algorithm}
\begin{enumerate}
    \item Start the program.
    \item Open the input C file in read mode and the output file in write mode.
    \item Initialize variables to track line numbers and multi-line comments.
    \item Read the source file line by line.
    \item Identify single-line and multi-line comments.
    \item Display the line numbers where comments are detected.
    \item Write only non-comment code into the output file.
    \item Repeat until the end of file is reached.
    \item Close all files.
    \item End the program.
\end{enumerate}
\newpage

\section*{Source Code}
\begin{lstlisting}[language=C]
#include <stdio.h>
#include <string.h>

int main() {
    FILE *in, *out;
    char line[300];
    int i, lineNo = 0, block = 0;
    in = fopen("input.c", "r");
    out = fopen("output.c", "w");
    if (!in || !out) return 1;
    
    while (fgets(line, sizeof(line), in)) {
        lineNo++; i = 0;

        if (!block && strstr(line, "//"))
            printf("Single-line comment at line %d\n", lineNo);

        while (line[i]) {

            if (!block && line[i]=='/' && line[i+1]=='*')
                block = 1, printf("Block comment starts at line %d\n", lineNo), i+=2;

            else if (block && line[i]=='*' && line[i+1]=='/')
                block = 0, printf("Block comment ends at line %d\n", lineNo), i+=2;

            else if (block || (line[i]=='/' && line[i+1]=='/'))
                break;

            else
                fputc(line[i++], out);
        }
        if (!block) fputc('\n', out);
    }
    fclose(in); fclose(out);
    return 0;
}
\end{lstlisting}

\newpage

\section*{Input}
The input file \texttt{input.c} contains a C program with single-line and multi-line comments.

\begin{lstlisting}[language=C]
#include <stdio.h>

// This is a single-line comment
int main() {
    /* This is a
       multi-line comment */
    printf("Hello World");
    return 0;
}
\end{lstlisting}

\section*{Output}
The output file \texttt{output.c} contains the same C program after removing all comments.

\begin{lstlisting}[language=C]
#include <stdio.h>

int main() {

    printf("Hello World");
    return 0;
}
\end{lstlisting}

\vspace{0.4cm}
\section*{Output Screenshot}

\begin{center}
    % Add terminal output screenshot here
    \includegraphics[width=0.85\textwidth]{ss.png}
\end{center}

\section*{Discussion}
This experiment demonstrates the working of a basic lexical analyzer used in compiler design. The program successfully detects both single-line and multi-line comments and displays their respective line numbers. By removing comments and generating a clean output file, the experiment highlights the importance of lexical analysis in source code processing. File handling and string manipulation concepts in C were effectively applied.

\end{document}
