\documentclass[12pt]{article}

% Page setup
\usepackage[a4paper, margin=1in]{geometry}

% Font
\usepackage{mathpazo}

% Formatting packages
\usepackage{setspace}
\usepackage{graphicx}
\usepackage{listings}
\usepackage{titlesec}

% Remove page numbers
\pagenumbering{gobble}

% Line spacing
\onehalfspacing

% Code style
\lstset{
    basicstyle=\ttfamily\small,
    frame=single,
    breaklines=true,
    tabsize=2
}

% Section formatting
\titleformat{\section}{\large\bfseries}{\thesection}{1em}{}

\begin{document}
\vspace{1cm}
\section*{Experiment No 01}
\section*{Experiment Name}
A program to read a C program from a source file and copy it into a text file.

\section*{Objective}
The objective of this experiment is to understand and implement file handling operations in C programming. This includes opening files, reading data from a source file, writing data into a destination file, and properly closing files. The experiment also helps in understanding character-wise file copying using standard input/output functions.

\section*{Algorithm}
\begin{enumerate}
    \item Start the program.
    \item Declare file pointers for source and destination files.
    \item Open the source file in read mode.
    \item Open the destination file in write mode.
    \item Check whether both files are opened successfully.
    \item Read characters from the source file one by one.
    \item Write each character into the destination file.
    \item Repeat the process until the end of the source file is reached.
    \item Close both the files.
    \item Display a success message.
    \item End the program.
\end{enumerate}
\newpage
\section*{Source Code}
\begin{lstlisting}[language=C]
#include <stdio.h>

int main() {
    FILE *source, *target;
    char ch;

    source = fopen("source.c", "r");
    target = fopen("output.txt", "w");

    if (source == NULL || target == NULL) {
        printf("Error opening file.\n");
        return 1;
    }

    while ((ch = fgetc(source)) != EOF) {
        fputc(ch, target);
    }

    printf("File copied successfully.\n");

    fclose(source);
    fclose(target);

    return 0;
}
\end{lstlisting}

\section*{Input}
A C source file named \texttt{source.c} containing a valid C program is provided as input to the program.
\newpage
\section*{Output}
The contents of the input file \texttt{source.c} are copied successfully into the output file \texttt{output.txt}.

\vspace{0.4cm}
\textbf{Output Screenshot:}

\begin{center}
    % Insert your screenshot image file below
    \includegraphics[width=0.85\textwidth]{test.png}
\end{center}

\section*{Discussion}
This experiment demonstrates the use of file handling concepts in C programming. Functions such as \texttt{fopen()}, \texttt{fgetc()}, \texttt{fputc()}, and \texttt{fclose()} were used to perform file operations. Character-by-character copying ensures accurate duplication of file contents. Proper error handling was implemented to check whether files are opened successfully. This experiment provides a strong foundation for understanding more advanced file operations in C.

\end{document}
